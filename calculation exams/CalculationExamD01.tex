\documentclass[10pt,answers,addpoints]{exam}
\usepackage{geometry}
\usepackage{amsmath,amsthm,amssymb,amsfonts,parskip,xypic}
\geometry{letterpaper}
\usepackage{graphicx}
\usepackage{color}
\usepackage{amssymb}
\usepackage{multicol}
\usepackage{epstopdf}
\DeclareGraphicsRule{.tif}{png}{.png}{`convert #1 `dirname #1`/`basename #1 .tif`.png}

\usepackage{mathpple}

\usepackage{coordsys}
\usepackage{pgf,tikz}
\usetikzlibrary{arrows}
\newcommand{\boxwidth}{5.25in}
%\pagestyle{empty}
%\setlength{\oddsidemargin}{.05in}
%\setlength{\evensidemargin}{-.50in}
%\setlength{\textwidth}{7in}
%\setlength{\topmargin}{-.250in}
%\setlength{\headheight}{0in}
%\setlength{\headsep}{.25in}
%\setlength{\topskip}{0in}
%\setlength{\textheight}{9.05in}
%\setlength{\parindent}{0in}
%\font\bigbf = cmbx10 scaled \magstep1
%\font\medbf = cmbx10 scaled \magstephalf

\input{153controls.tex}


\usepackage{pgfplots}
\usepgfplotslibrary{polar}
\pgfplotsset{compat=1.10}
\pgfplotsset{mypolarplot/.style={%
  clip=false, % needed for double line (last \addplot command)
  domain=0:360, % plot full cycle
  samples=180, % number of samples; can be locally adjusted
  grid=both, % display major and minor grids
  major grid style={black}, 
  minor x tick num=3, % 3 minor x ticks between majors
  minor y tick num=1, % 1 minor y tick between majors
  xtick={0,45,...,359},
  xticklabels={%
    $0$,
    $\frac{ \pi}{4}$,
    $\frac{ \pi}{2}$,
    $\frac{3\pi}{4}$,
    $\pi$,
    $\frac{5\pi}{4}$,
    $\frac{3\pi}{2}$,
    $\frac{7\pi}{4}$
  },
  yticklabel style={anchor=north}, % move label position
}}


%\shadedsolutions


% Include answers?
\noprintanswers
%\printanswers

%\bracketedpoints


% Where should the points be?
% Default
%\nopointsinmargin
% Left margin
\pointsinmargin
\marginpointname{ \points}
% Right margin
%\pointsinrightmargin






% Nice way to TeX sets
\def\set#1{\left\{ {#1} \right\}}
\def\setof#1#2{{\left\{#1\,|\,#2\right\}}}

\def\i{{\bold i}}

\def\j{{\bold j}}

\def\k{{\bold k}}

\def\u{{\bold u}}

\def\v{{\bold v}}

\def\w{{\bold w}}

\def\n{{\bold n}}

\def\d{{\partial}}


\def\C{{\mathbb C}}
\def\Z{{\mathbb Z}}
\def\F{{\mathbb F}}
\def\bF{{\mathbb F}}
\def\Q{{\mathbb Q}}
\def\R{{\mathbb R}}
\def\P{{\mathbb P}}
\def\N{{\mathbb N}}

\def\erf{{\text{erf}}}





\begin{document}

\runningheader{\bfseries Math ZZZ}{}{\bfseries Quiz \#X (Continued)}



\header{\bfseries\large Math 152}%\\Michael Janssen}
{\bfseries\large Calculation Exam D-01}
{}%{\includegraphics[height=.15in]{figures/CClicense.eps}}

%\makebox[\textwidth]{\hrulefill}
Name: \makebox[4.18in]{\hrulefill} \quad Score: \makebox[0.75in]{\hrulefill} % Section:\enspace\hrulefill}
%\makebox[\textwidth]{Name:\enspace\hrulefill}

%\vspace{0.3in}
%\makebox[\textwidth]{Section:\enspace\hrulefill}

\vspace{0.2in}

\textbf{Instructions: } There are 10 functions on this exam. Compute their derivatives using our derivative rules. You must completely correctly calculate at least 8 of the 10 derivatives in order to pass. You may not use a calculator, and you do not need to simplify! 




\begin{questions}

\question $f(x)=2x^3 - x^2 + 2$

\vfill

\question $g(y) = 3^y - \frac{1}{y}$

\vfill

\question $h(x) = (1-x)(x+e^x)$

\vfill


\question $k(w) = \dfrac{1+2w}{3-4w}$

\vfill

\question $p(t) = e^t \cos t $

\vfill

\newpage


\question $F(x) = \frac{\sec(x)}{2-\tan(x)}$

\vfill


\question $G(y) = (1+y+y^2)^{99}$

\vfill

\question $H(z) = \frac{1}{\sqrt{z^2-1}}$

\vfill

\question $K(w) = \arcsin(3w)-\arctan(w^2+1)$

\vfill

\question $P(t) = \left(\frac{t-1}{t^2+t+1}\right)^4$

\vfill

\thispagestyle{empty}

\end{questions}



\end{document}
