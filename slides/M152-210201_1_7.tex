\def\pgfsysdriver{pgfsys-dvipdfm.def}
%\documentclass[aspectratio=1610]{beamer}
\documentclass{beamer}
\usetheme[progressbar=head, titleformat=allcaps,block=fill]{metropolis}
\usepackage[orientation=landscape,size=custom,width=16,height=11.15,scale=.5,debug]{beamerposter}

%%% PACKAGES AND INPUTS
\usepackage{pgfopts}
\usepackage{pgfpages}
\usepackage{graphicx}
\usepackage{xcolor}
\usepackage{tikz}
\usetikzlibrary{3d,hyperref}
\usepackage{verbatim}
\usepackage{comment}
%\input{153controls}
\usepackage{pgfplots}
\usepackage{linalgjh}

\usepackage{multicol}


%%% POLAR SETUP
\usepgfplotslibrary{polar}
\pgfplotsset{compat=1.10}
\pgfplotsset{mypolarplot/.style={%
  clip=false, % needed for double line (last \addplot command)
  domain=0:360, % plot full cycle
  samples=180, % number of samples; can be locally adjusted
  grid=both, % display major and minor grids
  major grid style={black}, 
  minor x tick num=3, % 3 minor x ticks between majors
  minor y tick num=1, % 1 minor y tick between majors
  xtick={0,45,...,359},
  xticklabels={%
    $0$,
    $\frac{ \pi}{4}$,
    $\frac{ \pi}{2}$,
    $\frac{3\pi}{4}$,
    $\pi$,
    $\frac{5\pi}{4}$,
    $\frac{3\pi}{2}$,
    $\frac{7\pi}{4}$
  },
  yticklabel style={anchor=north}, % move label position
}}

%%%% COLORS
\definecolor{gold}{HTML}{ffb81c}
\definecolor{heypurple}{HTML}{6B0FFF}
\definecolor{darkgold}{HTML}{5A440D}
\definecolor{darkblue}{HTML}{00268D}
\definecolor{proof}{RGB}{55,54,172}
\definecolor{nicegreen}{RGB}{0,127,35}
\definecolor{remgrey}{RGB}{179,179,179}
\definecolor{Purple}{HTML}{3C0940}
\definecolor{darkgreen}{HTML}{214009}
\definecolor{Orange}{HTML}{7F2000}
\definecolor{Blue}{HTML}{003E78}
\definecolor{Gold}{HTML}{FFCB0A}
\definecolor{gvsublue1}{HTML}{0065a4}
\definecolor{gvsublue2}{HTML}{88B3DA}
\definecolor{unlred}{HTML}{D00000}
\definecolor{DordtGrey}{RGB}{127,127,127}
\definecolor{newred}{HTML}{85140D}
\definecolor{msugreen}{HTML}{023328}
\definecolor{seafoam}{HTML}{127F67}
\definecolor{mnmaroon}{HTML}{790018}
\definecolor{mngold}{HTML}{FFD75F}
\definecolor{fallfoliage}{HTML}{763626}
\definecolor{stone}{HTML}{336B87}
\definecolor{shadow}{HTML}{2A3132}
\definecolor{grass}{HTML}{486B00}
\definecolor{thundercloud}{HTML}{505160}
\definecolor{meadow}{HTML}{598234}
\definecolor{ink}{HTML}{20232A}
\definecolor{rubyred}{HTML}{A01D26}
\definecolor{stormysea}{HTML}{335252}
\definecolor{rust}{HTML}{AA4B41}
\definecolor{forest}{HTML}{1E434C}
\definecolor{crimson}{HTML}{8D230F}
\definecolor{vtgold}{HTML}{C99E10}
\definecolor{vtrust}{HTML}{9B4F0F}

%%%% BEAMER THEMES AND OPTIONS
%\usetheme[progressbar=head,block=fill]{metropolis}
%\setbeameroption{show notes on second screen}
%\metroset{titleformat=smallcaps,block=transparent}
%\usecolortheme{owl}
\setbeamercolor{progress bar}{fg=heypurple,bg=heypurple!30}
\setsansfont[BoldFont={Graphik Semibold},Numbers={Produkt}]{Graphik}


%%% Dordt
%\setbeamercolor{alerted text}{fg=gold}
%\setbeamercolor{normal text}{fg=black}
%%%\setbeamercolor{block}{transparent}
%%%\setbeamercolor{sectionpage}{bg=white}
%%%\setbeamercolor{titleformatpage}{bg=white}
%\setbeamercolor{frametitle}{fg=gold,bg=black!2}
%%%%% GREEN OPTION
%%%\setbeamercolor{alerted text}{fg=seafoam} 
%%%\setbeamercolor{frametitle}{bg=msugreen}



%%% Purple
\setbeamercolor{alerted text}{fg=heypurple}
\setbeamercolor{normal text}{fg=black}
%%\setbeamercolor{block}{transparent}
%%\setbeamercolor{sectionpage}{bg=white}
%%\setbeamercolor{titleformatpage}{bg=white}
\setbeamercolor{frametitle}{fg=heypurple,bg=black!2}
%%%% GREEN OPTION
%%\setbeamercolor{alerted text}{fg=seafoam} 
%%\setbeamercolor{frametitle}{bg=msugreen}

%%% UMICH OPTION
%\setbeamercolor{alerted text}{fg=gold}
%\setbeamercolor{frametitle}{bg=Blue}

%%% MN OPTION
%\setbeamercolor{alerted text}{fg=gold}
%\setbeamercolor{frametitle}{bg=mnmaroon}

%%% FALL MOUNTAIN
%\setbeamercolor{alerted text}{fg=fallfoliage}
%\setbeamercolor{frametitle}{bg=shadow}
%\setbeamercolor{alerted text}{fg=stone}
%\setbeamercolor{alerted text}{fg=grass}

%%% ICELAND
%\setbeamercolor{alerted text}{fg=meadow}
%\setbeamercolor{frametitle}{bg=thundercloud}


%%% VERMONT
%\setbeamercolor{alerted text}{fg=crimson}
%\setbeamercolor{alerted text}{fg=vtrust}
%\setbeamercolor{frametitle}{bg=thundercloud}


%%% INDUSTRIAL
%\setbeamercolor{alerted text}{fg=rubyred}
%\setbeamercolor{frametitle}{bg=ink}

%%% SAN FRANCISCO
%\setbeamercolor{alerted text}{fg=rust}
%\setbeamercolor{frametitle}{bg=stormysea}


%%%% MACROS
\def\rem#1{{\hfill \textit{\tiny {\color{remgrey} {#1}}}}}
\def\h#1{\alert{#1}}
\def\C{{\mathbb C}}
\def\Z{{\mathbb Z}}
\def\F{{\mathbb F}}
\def\bF{{\mathbb F}}
\def\Q{{\mathbb Q}}
\def\R{{\mathbb R}}
\def\P{{\mathbb P}}
\def\A{{\mathbb A}}
\def\N{{\mathbb N}}
\def\i{\mathbf i}}
\def\j{\mathbf j}}
\def\k{\mathbf k}}
\def\proj{{\text{proj}}}
\def\comp{{\text{comp}}}
\def\set#1{\left\{ {#1} \right\}}
\def\setof#1#2{{\left\{#1\,:\,#2\right\}}}


%%%% ENVIRONMENTS
\newenvironment{proof-idea}{\noindent{\alert{Proof Idea.}}\hspace*{1em}}{\qed\bigskip\\}
\newtheorem{conj}{Conjecture}
\newtheorem{prop}{Proposition}
\newtheorem{cor}{Corollary}
\newtheorem{defn}{Definition}
\newtheorem{question}{Question}
\newtheorem{goal}{Goal}

\usepackage{cancel}
\newcommand\lheq{\mathrel{\overset{\makebox[0pt]{\mbox{\normalfont\tiny\sffamily \alert{L'H}}}}{=}}}

\title{\S 1.7: Limits, Continuity, and Differentiability}
%\subtitle{Welcome!}
\author{Dr.\ Mike Janssen}
\date{February 1, 2021}

\begin{document}

\frame{\titlepage}


\frame{
	\frametitle{Preview Activity Discussion}

}


\frame{
	\frametitle{Left/Righthand limits}
	
	One way a limit can fail to exist at a point is if the function seems to approach two different numbers depending on the side we're coming from.\pause 
	
	\begin{definition}
We say that $f$ has limit $L_1$ as \textbf{$x$ approaches $a$ from the left} and write
\[
\lim\limits_{x\to a^-} f(x) = L_1
\]
provided we can make the value of $f(x)$ as close as we like to $L_1$ by taking $x$ sufficiently close to $a$ while always having $x < a$. We call $L_1$ the \textbf{left-hand limit} of $f$ as $x$ approaches $a$.\pause\ Similarly, we say that $L_2$ is the \textbf{right-hand limit} of $f$ as $x$ approaches $a$ and write
\[
\lim\limits_{x\to a^+} f(x) = L_2.
\]
\end{definition}





}


\frame{
	\frametitle{Example/Big Important Fact}
	
	\textbf{Example:}
	
	\vspace{-.25in}
	
%	\begin{picture}(0,0)(-200,110)
%\rescaleby{1}{10}{\hlabel} 
%\rescaleby{1}{10}{\vlabel} 
%\coordgrid(-100,-100)(100,100)
%\coordsys(-100,-100)(100,100)
%\put(100,0){\sethlabel{x}}
%\put(0,100){\setvlabel{y}}
%\gridstyle{\linethickness{0.05pt}}{\linethickness{0.14pt}}
%\end{picture}
	
	%\vspace{3in}\pause
	
	 \begin{center}   
	\includegraphics[width=3in]{img/blank-grid.png}  
\end{center}

	
	
	\textbf{Fact:} $\lim\limits_{x\to a} f(x) = L$ if and only if $\lim\limits_{x\to a^+} f(x) = \lim\limits_{x\to a^-} f(x) = L$

}


\frame{
	\frametitle{Activity 1.7.2}
	
\pause
	
%	\begin{picture}(0,0)(-200,110)
%\rescaleby{1}{25}{\hlabel} 
%\rescaleby{1}{25}{\vlabel} 
%\coordgrid(-100,-100)(100,100)
%\coordsys(-100,-100)(100,100)
%\put(100,0){\sethlabel{x}}
%\put(0,100){\setvlabel{y}}
%\gridstyle{\linethickness{0.05pt}}{\linethickness{0.14pt}}
%\end{picture}
	
	\begin{center}
	\includegraphics[width=3in]{img/1_7_fcn.png}
	\end{center}

}

\frame{
	\frametitle{Continuity}
	
	\begin{definition}
A function $f$ is \textbf{continuous at $x=a$} provided that

\begin{enumerate}
	\item $f$ has a limit as $x\to a$,
	\item $f$ is defined at $x=a$, and
	\item $\lim\limits_{x\to a} f(x) = f(a)$
\end{enumerate}
\end{definition}

\pause

What does this mean? \rem{cont. implies $\lim = $ fcn}

}

\frame{
	\frametitle{Activity 1.7.3}

\begin{center}
		\includegraphics[width=3in]{img/1_7_fcn.png}
	\end{center}
	
}

\frame{
	\frametitle{The relationship between differentiability and continuity}
	
	\textbf{Fact:} We have seen that a function $f(x)$ can be continuous at $x=a$ but fail to have $f'(a)$ exist. That is, a function can be continuous without being differentiable.
	
	What was that function $f(x)$?
	
	\pause
	
	\textbf{Question:} Can a function be differentiable without being continuous?
	
	\pause
	
	\textbf{No.}

}


\frame{
	\frametitle{Activity 1.7.4}



\begin{center}
		\includegraphics[width=3in]{img/1_7_fcn.png}
	\end{center}
	


}


\frame{
	\frametitle{Reminders}
	
	\begin{itemize}
	\item 
\end{itemize}

}




\end{document}